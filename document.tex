\documentclass[conference]{IEEEtran}


\title{Entrega Inicial Teoría de Grafos}
\author{
	\IEEEauthorblockN{Juan Sebastian Pedraza Guevara }
	\IEEEauthorblockA{\textit{Escuela de Ingeniería Ciencia y Tecnología} \\
		\textit{Matemáticas Aplicadas y Ciencias de la Computación}\\
		\textit{Universidad del Rosario}\\
		Bogotá , Colombia\\
	}
	
	
}
\begin{document}
	
	\maketitle
	
	
	\section{Introducción}
	En este proyecto, exploramos cómo los grafos pueden usarse para analizar, modelar y resolver laberintos. Transformamos los laberintos en estructuras de grafos y aplicamos algoritmos de búsqueda y recorrido para encontrar rutas óptimas o soluciones. Este enfoque tiene aplicaciones en planificación de rutas, robótica y simulación de entornos complejos. 
	\\ 
	\\
	A lo largo del proyecto, presentamos conceptos clave, algoritmos seleccionados y ejemplos de aplicación en diferentes  laberintos. La intersección entre la teoría de grafos y la resolución de laberintos destaca la eficacia de la teoría de grafos en la solución de problemas computacionales prácticos.
	
	\section{Descripción del problema}
	Este problema aborda la resolución eficiente de laberintos, donde se busca encontrar una ruta desde un punto de inicio hasta un destino a través de pasillos. La teoría de grafos ofrece un enfoque estructurado para transformar laberintos en estructuras de grafos. Al aplicar algoritmos de búsqueda y recorrido, como la búsqueda en profundidad o en anchura, se buscan rutas óptimas o soluciones para navegar en laberintos complejos. Este enfoque tiene aplicaciones en planificación de rutas, robótica y simulación de entornos complejos, demostrando la eficacia de los conceptos de grafos en la resolución de problemas prácticos.
	\section{Objetivo general y objetivos específicos}
	\subsection{Objetivo Especifico}
	El objetivo principal de este proyecto es aplicar conceptos de teoría de grafos para analizar, modelar y resolver problemas de laberintos. Se busca implementar algoritmos de búsqueda y recorrido en estructuras de grafos generadas a partir de laberintos, con el fin de encontrar rutas óptimas o soluciones eficientes en diferentes contextos.
	\\
	\\
	\subsection{Objetivos Específicos}
	\vspace{0.3cm}
	\begin{enumerate}
		\item Analizar la representación de laberintos como grafos y comprender cómo los nodos y aristas reflejan los elementos del laberinto.
		
		\item Seleccionar y aplicar algoritmos de búsqueda y recorrido de grafos, como búsqueda en profundidad y búsqueda en anchura, para encontrar rutas desde el punto de inicio al destino en laberintos.
		
		\item Diseñar y programar una herramienta computacional que tome laberintos como entrada y genere rutas solucionadas utilizando conceptos de teoría de grafos.
		
		\item Evaluar y comparar la eficacia y eficiencia de los algoritmos implementados en diferentes tipos de laberintos, considerando factores como el tamaño, complejidad y características.
		
		\item Documentar el proceso de implementación, los resultados obtenidos y las conclusiones alcanzadas, resaltando la relevancia de la teoría de grafos en la resolución de problemas de laberintos y su utilidad en el ámbito computacional.
	\end{enumerate}
	
	
	\section{Bibliografía}
	Ahuja, R. K., Magnanti, T. L., Orlin, J. B. (1993). Network Flows: Theory, Algorithms, and Applications. Prentice Hall. \\
	\\
	Douglas B. West, introduction to graph theory, 2nd edition. Pearson, 2000.\\
	\\
	K. Rosen, Discrete Mathematics and its Applications, 7th Edition. McGraw Hill, 2012.\\
	\\
	Douglas B. West, introduction to graph theory, 2nd edition. Pearson, 2000.
	
	
	
\end{document}
